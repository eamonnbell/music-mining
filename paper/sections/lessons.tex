%!TEX root = ../ycac2vec.tex
\subsection{Harmonic function, polysemy and word embedding models}
Word embedding models have been shown to perform well on analogical reasoning tasks. If an embedding model is trained on a sufficiently large and informative natural language corpus, the model can be used to solve for $x$ in analogies of the following form:
$$\textrm{man}:\textrm{king}::\textrm{woman}:x$$
By performing simple vector operations on the vectors corresponding to the known terms in the analogy, we end up with a new vector that represents $x$, which turns out to be closest to the vector in the space that corresponds to the token `$\textrm{queen}$.' 
Turning to the musical case, consider a chain of major triads which are related by a perfect fifth. The language of functional harmony describes the relation between any two adjacent members of this chain (from left to right) as the `dominant-of' relation. For example:
$$\hdots \textrm{G major} \xrightarrow{\textrm{dominant of}} \textrm{C major} \xrightarrow{\textrm{dominant of}} \textrm{F major} \hdots$$
In virtue of the common kind of relation between adjacent pairs in this sequence, we can rewrite the relationship between the major triads in a form corresponding to the analogical reasoning question:
$$\hdots \textrm{G major} : \textrm{C major}  :: \textrm{C major} : \textrm{F major} \hdots$$
This form captures a notion fundamental to functional harmony that the C major triad is multivalent. It can operate as both the subject and object of the `dominant-of' relation. 

We observe a parallel here with the problem of polysemy in natural language processing. Just as C major has one harmonic function in one context (it is the dominant in passages in the key of F major), and another function in a different context (it is not the dominant in passages of C major), so too do many words have different meanings in different contexts.

Our results suggest that the distribution of musical tokens in the embedding space that participate in these overlapping analogical chains described above is topologically regular. By analogy, we conjecture that polysemous tokens in natural language are similarly structured in the neighborhoods of their positions in well-trained, plausible word embedding spaces.

\subsection{Embedding models for stylistic analysis}
Mathematical models of musical harmony such as that offered by \cite{Callender2008} tend to situate major triads as discrete nodes in an network embedded a geometrically regular space. Edges between nodes can be determined either by regularities in transpositional equivalence between nodes or by regularities in voice-leading distance between adjacent nodes. Observed musical works can be viewed as traversing some path in this space. 

These models can be extended to incorporate pitch sets smaller and larger than triads, and to cope with pitch multisets. \cite{Quinn2006, Yust2015} expand on early work from \cite{Lewin1959} on characterizing pitch-class sets by their Discrete Fourier Transform to provide a set of spaces in which pitch sets of different cardinality can be modeled.

In all these cases, the harmonic spaces proposed are a \textit{a priori} constructions. Though historically grounded and based on the musical intuitions of theorists, these spaces are developed on the basis of fundamental analysis of properties of pitch sets in the asbtract.

By contrast, word embedding models allow the analyst to construct a space that is learned \textit{a posteriori}, based on a sequences observations from a corpus. For example, training an embedding model on an representative corpus of Bach's works results in harmonic space that captures some sense of the relationship between harmonies as they appear in the output of that composer. Alternatively, \cite{Wolf2014} points an applications of word embeddings in machine translation that might be used to explore two harmonic styles in one embedding space.

Furthermore, by comparing the structure of embedding spaces trained on the works of different composers, we can develop a discourse of comparative stylistic analysis which is grounded in the particularities of their particular outputs. On this view, observed works are now viewed as paths on a space that is contextual and composer- or style-dependent, rather than on an \textit{a priori} space preferred for its historical resonances or concise topological characterization.
\subsection{Corpus}\label{corpus}

The Yale/Classical Archives Corpus (YCAC) is a database of pitch-class
and time data from MIDI files contributed by users of
classicalarchives.com encoding 8,980 distinct pieces of music
\cite{white2014yale}. Each piece is represented both by a sequence of
time-coded music21.Chord objects and additional local key estimations, which were obtained by ``salami slicing'' the original MIDI file. That is, a new chord token is created at every moment a voice enters or leaves the musical texture.  A large subset of the corpus was divded into six subcorpora based on the dates of composition of the pieces. Where an exact date of composition was not found in the piece metadata provided with the YCAC, the midpoint of the date range given in the metadata was used.\footnote{Typically, this range was the lifespan of the composer.}

\subsection{Embedding space}\label{embedding-space}

A word2vec algorithm was used to create a number of word-embedding
spaces for the entire corpus and for corpora consisting only of the
works of a single
composer.\footnote{The implementation used was the word2vec model provided by the Python module `gensim`, which uses a skip-gram negative sampling (SGNS) model which has been shown to be effective on large textual corpora. \cite{rehurek_lrec}}.
The algorithm treats each chroma vector as a word in a sentence. It
returns an n-dimensional real-valued vector for each word. t-SNE
dimensionality reduction was applied to the resultant word-embedding
space to demonstrate its plausbility. PCA was applied to the resultant
word-embedding space, and the locations of chroma vectors corresponding
to major triads were plotted.

%!TEX root = ../ycac2vec.tex

We apply a word embedding model to a large symbolic corpus of classical music to learn an embedding space where chords are represented by real-valued vectors.
In early classical music, the first two principal components of the embeddings of major triads form a circle. In music from later composers, this circular topology is less evident. 
Remarkably, the order in which major triads are arranged on this structure corresponds to their order in the circle of fifths.
The emergence of this structure is justified by reasoning about the probabilistic embedding model and the composition of classical music.
Deviation from the circle of fifths in the embedding space of music from later composers corresponds to intutions from music theory about the decline in use of functional harmony.
We situate our results in the context of current statistical research into functional harmony in common practice music. We make a connection with the problem of polysemy in natural language processing by pointing out that major chords in classical music can be viewed as having multiple meanings. This motivates a similar topological anlysis of both music and natural language in service of disambiguating multiple meanings.
We show how this technique can be used for large-scale, quantitative stylistic analysis of music.
